\documentclass[a4paper]{article}

% Packages
\usepackage[margin=1.9cm]{geometry} % margin layout
% parskip also adds a bit of space after each paragrah
\usepackage{parskip} % needed for un-doing paragraph indentation
%\usepackage{titlesec}

% Custom commands
\newcommand{\sectionSpace}{\vspace{1em}} % custom section break


%\setlength{\parindent}{0pt} % another way to have 0 paragraph indent
% \par for newlines works as well


\begin{document}
\title{INTENTIONALLY SCUFFED NOTES}
\author{Several Authors}
\maketitle
\newpage

%\normalsize % reset \Large{} text. Otherwise, put this into a frame


\begin{center}
    \Large 
    \textbf{
        REMEMBER THAT AT ANY POINT, THESE EQUATIONS, FORMULAE AND NOTES IN GENERAL MAY CONTAIN INTENTIONAL ERRORS. WE ARE NOT LIABLE FOR ANYTHING} 
\end{center}

\sectionSpace
\section{Introduction}
    Edit later.



\sectionSpace
\section{Proof by Induction}
    \subsection{Sequences and Series}



\sectionSpace
\section{Functions}
Let us focus on functions. 
    \subsection{Linear equation}
    A linear equation is a polynomial degree 1. It can be shown in several ways.
    \begin{itemize}
        \item General form: $ax + by = c$
        \item Slope-intercept form: $y = mx + c$
    \end{itemize}


\sectionSpace
\section{Trigonometry}
    \subsection{Angle Sum and Difference}
    $$\sin(A \pm B) = \sin A \cos B \pm \cos A \sin B$$
    $$\cos(A \pm B) = \cos A \cos B \pm \sin A \sin B$$

    \subsection{Double Angle Identities}
    Double angle identities are derived from the angle and sum difference equations. 

    We know that
    $$\sin(2x) = 2\sin(x)\cos(x)$$
    therefore, we can say that
    $$\sin(4x) = 4\sin^2(x)\cos^2(x)$$





\sectionSpace
\section{Calculus of the Differential Kind}

\sectionSpace
\section{Integration}



\sectionSpace
\section{Complex Numbers}
    \subsection{Euler's Form}
    We know that
    $$\sin^\theta + \cos^2\theta = 1$$
    We also know the famous equation known as Euler's formula. The imaginary number one. With the pi. No, not the one with polyhedrons. To be more precise, we mean Euler's identity. The "beautiful" equation.
    $$e^{i\pi} + 1 = 0$$

    Therefore, we can then conclude that:
    $$e^{i\theta} + \sin^2\theta + \cos^2\theta = 0$$

    $$\frac{e^{i\theta} + \sin^2\theta}{-\cos^2\theta} = 1$$
    $$-\frac{e^{i\theta}}{\cos^2\theta} -\frac{\sin^2\theta}{\cos^2\theta} = 1$$
    $$-e^{i\theta}\sec^2\theta -\tan^2\theta = 1$$
    $$-e^{i\theta}\sec^2\theta = 1 + \tan^2\theta$$
    $$-e^{i\theta}\sec^2\theta = \sec^2\theta$$
    $$-e^{i\theta} = 1 \Rightarrow e^{i\theta} = -1$$

    What a beautiful conclusion!

\sectionSpace
\section{Linear Algebruh}


\end{document}
